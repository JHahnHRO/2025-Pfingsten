\subsection{Epistemologie}

Epistemologie ist die philosophische Disziplin, die nach der Natur von \enquote{Wissen} und \enquote{Wahrheit}
fragt. Sie stellt Fragen wie \enquote{Was ist Wissen?}, \enquote{Was ist Wahrheit?},
\enquote{Wie finde ich die Wahrheit?}.

Es gibt viele verschiedene philosophische Ansätze, sich diesen Fragen zu nähern. Für die Wissenschaftstheorie ist vor allem einer von
Bedeutung: Der empirische Ansatz.

Empirismus stellt als Mittel der Erkenntnis-Gewinnung das Experiment an der realen Welt in den Vordergrund.

Dass wir Empirismus als valide einschätzen, ergibt
sich aus unseren metaphysischen Grundannahmen:
\begin{enumerate}
    \item Weil die Welt Naturgesetzen folgt, die beobachtbare Konsequenzen haben,
    und Beobachtungen mit unseren Sinnen uns tatsächliche Informationen über die Welt geben, können wir durch
    sorgfältige Beobachtung der Natur Rückschlüsse ziehen auf die Naturgesetze, die gelten müssen.
    \item Weil wir transitive Schlussfolgerungen zulassen, können wir auch Messgeräte zu Hilfe ziehen, um Experimente durchzuführen. Die Beobachtung eines ausschlagenden Zeigers kann eine ebenso aussagekräftige Beobachtung der Natur sein, wie die direkte Beobachtung.
\end{enumerate}
