% !TeX spellcheck = de_DE
\documentclass[fontsize=11pt,fleqn,a4paper]{scrartcl}

\author{Johannes Hahn \and Lena-Elisa Schneegans}
\title{Eine Erkundung der Wissenschaftstheorie}
%\subtitle{}
\date{06.06.25 -- 09.06.25}


%%% General math packages

% AMS
\usepackage{amsmath,  
            amssymb,  % Symbols
            amsthm,   % provides theorem environments
            amsfonts  % fonts like \mathbb and \mathfrak
            }

% useful things for math typesetting like \smash, \psmallmatrix, ...
%\usepackage{mathtools}

% For \Set and \set. Automatic resizing of the curly braces and the middle vertical line
%\usepackage{braket}

% For even more extensible arrows
%\usepackage{extpfeil}

% Blockmatrices
%\usepackage{multirow}

% and brakets around matrix rows
%\usepackage{bigdelim}

% for \mathscr
%\usepackage{mathrsfs}
%\input{_preamble/math_symbols}
%%% Theorem environments

% Define name strings
\newcommand{\captionstringtheorem}{Theorem}
\newcommand{\captionstringcentralquestion}{Central Question}
\newcommand{\captionstringmaintheorem}{Main Theorem}
\newcommand{\captionstringlemma}{Lemma}
\newcommand{\captionstringcorollary}{Corollary}
\newcommand{\captionstringlemmadef}{Lemma and definition}
\newcommand{\captionstringtheoremdef}{Theorem and definition}
\newcommand{\captionstringdefinition}{Definition}
\newcommand{\captionstringproposition}{Proposition}
\newcommand{\captionstringexample}{Example}
\newcommand{\captionstringconjecture}{Conjecture}
\newcommand{\captionstringconvention}{Convention}
\newcommand{\captionstringremark}{Remark}


% Definition of two similar styles
\newtheoremstyle{dotless} % Name
			{\bigskipamount}    % Space above
			{\bigskipamount}    % Space below
			{\nopagebreak}      % Body font, also suppress pagebreak between "Theorem 3.14:" and text
			{}                  % Indent amount
			{\bfseries}         % Theorem head font
			{:}                 % Punctuation after theorem head
			{\newline}          % Space after theorem head
			{}                  % Theorem head spec (can be left empty, meaning 'normal')
\newtheoremstyle{dotless2} % Name
			{\bigskipamount}    % Space above
			{0.0em}             % Space below
			{}                  % Body font
			{}                  % Indent amount
			{\bfseries}         % Theorem head font
			{:}                 % Punctuation after theorem head
			{0.5em}             % Space after theorem head
			{}                  % Theorem head spec (can be left empty, meaning 'normal')
% "3.14 Theorem" instead of "Theorem 3.14"
\swapnumbers

\newcounter{theoremnumber}
\numberwithin{theoremnumber}{section}

\theoremstyle{dotless}
\newtheorem{theorem}[theoremnumber]{\captionstringtheorem}
\newtheorem{maintheorem}[theoremnumber]{\captionstringmaintheorem}
\newtheorem{centralquestion}[theoremnumber]{\captionstringcentralquestion}
\newtheorem{theoremdef}[theoremnumber]{\captionstringtheoremdef}
\newtheorem{lemma}[theoremnumber]{\captionstringlemma}
\newtheorem{lemmadef}[theoremnumber]{\captionstringlemmadef}
\newtheorem{corollary}[theoremnumber]{\captionstringcorollary}
\newtheorem{proposition}[theoremnumber]{\captionstringproposition}
\newtheorem{definition}[theoremnumber]{\captionstringdefinition}
\newtheorem{example}[theoremnumber]{\captionstringexample}
\newtheorem{conjecture}[theoremnumber]{\captionstringconjecture}

\newtheorem*{convention}{\captionstringconvention}

\theoremstyle{dotless2}
\newtheorem{remark}[theoremnumber]{}


%\input{_preamble/math_styles}
%%% Everything tikz

\usepackage{tikz}
\usetikzlibrary{arrows}
\usetikzlibrary{cd}
\usetikzlibrary{babel}

%\input{_preamble/algorithms.tex}
%%% Tables and figures
\numberwithin{table}{section}
\numberwithin{figure}{section}
\usepackage{graphicx}

\usepackage{tabularx,ragged2e}
\newcolumntype{C}{>{\Centering\arraybackslash}X} % centered "X" column
\newcolumntype{L}{>{\Centering\arraybackslash}{\hsize=.5\hsize}X} % centered "X" column
\newcolumntype{S}{>{\Centering\arraybackslash}{\hsize=.5\hsize}X} % centered "X" column



%%% Language preamble for german

% Language itself
\usepackage[ngerman]{babel}
% Font encoding to represent umlauts correct in PDFs documents instead of combining them from other
% characters like "u instead of ü. Do not use with XeLaTeX or LuaTeX
\usepackage[T1]{fontenc}
% Encoding of the source code. Do not use with XeLaTeX or LuaTeX
\usepackage[utf8]{inputenc}


% Language specific strings
%   Names of theorem environments

\addto\captionsngerman{
	% \see command from makeidx
%	\@ifpackageloaded{makeidx}{
%		\renewcommand{\seename}{siehe}
%	}{}
	\renewcommand{\captionstringtheorem}{Satz}
	\renewcommand{\captionstringmaintheorem}{Hauptsatz}
	\renewcommand{\captionstringcentralquestion}{Zentrale Fragestellung}	
	\renewcommand{\captionstringlemma}{Lemma}
	\renewcommand{\captionstringcorollary}{Korollar}
	\renewcommand{\captionstringlemmadef}{Lemma und Definition}
	\renewcommand{\captionstringtheoremdef}{Satz und Definition}
	\renewcommand{\captionstringdefinition}{Definition}
	\renewcommand{\captionstringproposition}{Proposition}
	\renewcommand{\captionstringexample}{Beispiel}
	\renewcommand{\captionstringconjecture}{Vermutung}
	\renewcommand{\captionstringconvention}{Vereinbarung}
	\renewcommand{\captionstringremark}{Bemerkung}
}
\makeatletter
%\@ifpackageloaded{biblatex}{
%\DefineBibliographyStrings{german}{
%	bibliography = {Bibliographie},
%	references = {Referenzen}
%}
%}{}
\@ifpackageloaded{exframe}{
	\exerciseconfig{termsheet}{Blatt}
	\exerciseconfig{termsheets}{Blätter}
	\exerciseconfig{termproblem}{Aufgabe}
	\exerciseconfig{termproblems}{Aufgaben}
	\exerciseconfig{termsolution}{Lösung}
	\exerciseconfig{termsolutions}{Lösungen}
	\exerciseconfig{termpoint}{Punkt}
	\exerciseconfig{termpoints}{Punkte}
}
\makeatother

%%% Text styles

% For the interrobang
\usepackage{textcomp}

% Additional underlining options, especially dotted underlining, line breaks in underlined text etc.
\usepackage[normalem]{ulem} % option not to change look of \emph{}
\newcommand*{\udot}{\dotuline}
\AtBeginEnvironment{definition}{\renewcommand{\emph}{\udot}}
\AtBeginEnvironment{lemmadef}{\renewcommand{\emph}{\udot}}
\AtBeginEnvironment{theoremdef}{\renewcommand{\emph}{\udot}}


% Skip lengths
\setlength{\parindent}{0em}
\setlength{\parskip}{0em}


% Pimp enumerate and itemize environments. More counter options and resuming numbering
\usepackage{enumitem}

% Style of enumerations and itemizations
\renewcommand{\labelenumi}{\alph{enumi}.)}  % Counter enumi wird immer als a.) b.) c.) dargestellt.
\renewcommand{\labelenumii}{\roman{enumii}.)}  % Counter enumii wird immer als i.) ii.) iii.) dargestellt.


\usepackage{csquotes}

%\usepackage{multicol}

\usepackage{siunitx}


% Für schönere Tabellen
\usepackage{booktabs}
\usepackage{array}

% For colorful notes in the margin
% Load after tikz!
\usepackage[backgroundcolor=red!20,textsize=footnotesize]{todonotes}
\newcommand{\progressMarker}[1]{\todo[backgroundcolor=green!20,linecolor=green]{#1}}

%\input{_preamble/bibtex_only.tex}
%\input{_preamble/biblatex_bibtex.tex}
%%%% Bibliography with biblatex + bibtex

\usepackage[
	style=alphabetic,
	sorting=nty,
	url=false,
	natbib=true,
	backend=biber,  % Biber backend
	defernumbers=true
]{biblatex}
% The .bib-file(s) for this document
\addbibresource{wissenschaftstheorie.bib}



% !! Hyperref before imakeidx !!
%%% PDF stuff
%%%
%%% Include hyperref before imakeidx !!

\usepackage[
	pdfpagelabels=true,
	plainpages=false
	]{hyperref}
	
\hypersetup{
colorlinks=true,
linkcolor=red,
urlcolor=blue,
citecolor=blue,
linktocpage=true, % Page numbers will be the links in t.o.c instead of the headings themselves
pdfpagelayout={OneColumn},
pdfstartview= % empty to cause the viewer to use its preferred behaviour instead of dictating a behaviour at opening of the document
}

\makeatletter
{\hypersetup{
pdfinfo=
	{  
		Title={\@title},
		Author={Johannes Hahn, Andrea Hanke},
		Keywords={Irreducible representations, tensors, spherical harmonics, Gelfand-Tsetlin basis},
		Subject={Representation theory}
	}
}}
\makeatother
%%% Index

% for multiple indices
\usepackage{imakeidx}
\indexsetup{
	level=\chapter*,                % Headings of indices will be displayed as \chapter* would
	toclevel=chapter,               % Indices will appear in the table of contents like a chapter
	headers={\indexname}{\indexname}% Header will contain the name
	}
% creates a new index
\makeindex[
	% internal name of the index
	name=terms,
	% Displayed name
	title=Stichwortverzeichnis,
	% Will be listed in the table of contents
	intoc,
	% will be displayed in two columns
	columns=2,
	% will be generated automatically with makeindex and the given program options inlcuding the
	% *.ist file (index style) that is to be used.
	program=makeindex,options=-s terms.ist
	]



%\input{_preamble/uebungen}


\begin{document}

\maketitle

\tableofcontents
\pagebreak


\section{Was ist Wissenschaft? - Ein Überblick über eine Einführung in einen Exkurs der Wissenschaftsphilosophie}


Was ist Wissenschaft? Was unterscheidet sie von Nicht-Wissenschaft und Pseudo-Wissenschaft? Was ist die
\enquote{wissenschaftliche Methode}? Ist Mathematik eine Wissenschaft? Informatik? Medizin? Wie erkenne ich das?

Diese und viele weitere Fragen wollen wir klären. Wir werden uns dafür zunächst kurz mit ein bisschen mit den
philosophischen Grundlagen auseinandersetzen.
% !TeX root = wissenschaftstheorie.tex
% !TeX spellcheck = de_DE

\subsection{Metaphysik}

Metaphysik bezeichnet die philosophische Disziplin, die ganz allgemein nach der Grundstruktur der Realität fragt.\cite{sep-metaphysics}
Zur Metaphysik gehören Fragen wie \enquote{Leben wir in der Matrix?},
\enquote{Existiert überhaupt irgendetwas und, wenn ja, warum?}, \enquote{Gibt es Gott?}, \enquote{Gibt es freien Willen?} uvm.

Das Ziel aller Wissenschaft\footnote{Eigentlich aller Versuche der Erkenntnisgewinnung,
    welche die Menschheit je unternommen hat. Aber wir wollen bei unserem eigentlichen Thema bleiben.}
ist es, die Realität zu verstehen. Die Metaphysik hilft uns dabei, bei diesem Unterfangen nicht ganz von vorne anfangen
zu müssen, sondern stellt uns Begrifflichkeiten bereit, auf denen wir aufbauen können, um zu verstehen wie
Wissenschaftler (im Allgemeinen) die Welt sehen.

\medskip

Gewisse metaphysische Grundannahmen werden in der Wissenschaft normalerweise nicht hinterfragt und in allem Tun und
Denken axiomatisch vorausgesetzt. Wir wollen das auch tun, aber zumindest wollen wir ein paar dieser Annahmen, die sonst
implizit bleiben, jetzt explizit nennen:

\subsubsection{Realismus I - Die Welt existiert}

Dies steht im Gegensatz zur Idee des Solipsismus. Nach René Decartes'\footnote{franz. Philosoph und Mathematiker, 1596 –- 1650} berühmten \enquote{Ich denke, also bin ich}
-Argument, kann sich jeder von uns sicher sein, dass man selbst existiert, aber es ist schwer absolute Sicherheit
darüber zu gewinnen, dass noch irgendjemand oder irgendetwas anderes wirklich existiert\footnote
{Descartes war der Meinung,
    auch dafür wasserfeste Argumente gefunden zu haben; der Rest ist der Philosophen ist nicht so sehr davon
    überzeugt wie von \enquote{Ich denke, also bin ich}}. Solipsismus ist, stark vereinfacht, die Idee, dass alles, wirklich alles andere Einbildung des eigenen Geistes ist.
\footnote
{Das ist kein Standpunkt, den irgendjemand ernsthaft vertritt, sondern eher eine denkbare Alternative,
    die man eigentlich gerne ausschließen möchte, aber es mangels guter Argument nicht unbedingt kann.}

Die Wissenschaft lehnt dies ab und geht von einer real existierenden Welt aus, die es zu untersuchen gilt. Offenbar
eine sehr nützliche Annahme, weil sonst kein Untersuchungsgegenstand für Wissenschaftler existierte.

\subsubsection{Realismus II - Die Welt ist wahrnehmbar}

Dies ist im Wesentlichen die Annahme, dass wir nicht in der Matrix\footnote
{Wie haben Philosophen sich nur früher Dinge erklärt,
    bevor philosophische Ideen verfilmt wurden und allgemeines Kulturgut wurden!?}
leben: Die Welt existiert nicht nur; unsere Sinne nehmen das, was existiert, zumindest teilweise wahr.

Man beachte, dass niemand (mehr) postuliert, dass unsere Sinne eine fehlerfreie oder vollständige Erfassung der Welt
ermöglichen; ganz im Gegenteil. Aber es wird eben zumindest davon ausgegangen, dass in den allermeisten Fällen
Dinge, die wir sehen und anfassen können, auch tatsächlich existieren. Vielleicht existiert noch mehr als das, aber
\emph{mindestens} das existiert schon einmal.

\subsubsection{Realismus III - Die Welt ist \enquote{objektiv}}

Das Wort \enquote{objektiv}
hat viele (mehr oder weniger) verschiedene Bedeutungen in verschiedenen Kontexten. Hier meinen wir damit das, was
man auch als \enquote{Intersubjektivität}
bezeichnen könnte, also die Annahme, dass für alle anderen dieselbe Welt existiert, die für einen selbst existiert.
Anders formuliert könnte man dies auch als eine noch stärkere Ablehnung des Solipsismus verstehen: Ich lebe nicht in meiner
eigenen Welt, die nur für mich existiert; die Welt existiert auch unabhängig von mir.

Unter Objektivität kann man auch eine gewisse \enquote{Neutralität} der Welt gegenüber dem Wahrnehmenden verstehen: Wenn ich in der exakt gleichen Situation wäre wie du, dann würde ich exakt das wahrnehmen wie du. Nur, weil ich eben ich bin, gelten nicht plötzlich andere Regeln für mich als für dich.

\subsubsection{Regularität \& Kausalität - Die Natur folgt Gesetzen}

Dies ist teilweise eine überprüfbare, beobachtbare Behauptung, teilweise aber auch eine Grundannahme. Wir fangen mit
der Wissenschaft nicht von vorne an; wir wissen bereits etliches, was unsere Vorfahren herausgefunden haben und das
allermeiste davon folgt auch irgendwelchen Gesetzmäßigkeiten. Wissenschaftler haben darüber hinausgehend aber auch
die Annahme, dass alles andere ebenfalls gesetzmäßig geschieht. Möglicherweise kennen wir noch nicht alle
Naturgesetze, aber dass es sie gibt und dass alles, was passiert, von irgendeinem dieser Gesetze erfasst wird, sind
Annahmen, die (fast) alle Wissenschaftler voraussetzen.

Darunter fällt auch die Annahme, dass Naturgesetze sich nicht spontan ändern, sondern räumlich und zeitlich konstant sind. (Und
wenn sie es einmal nicht sind, dann ist die Änderung ein Ausdruck eines noch grundlegenderen Naturgesetzes).

Noch etwas grundlegender ist darin auch die Annahme versteckt, dass die Natur dem Prinzip von Ursache und Wirkung folgt:
Nichts geschieht einfach so; alles hat irgendeine Ursache.\footnote{Nein,
    wir diskutieren jetzt nicht über Determinismus und Quantenphysik. So viel Zeit haben wir nicht. Es sei nur so viel
    gesagt, dass die Quantenmechanik auch eine deterministische Theorie ist, wenn man sich Mühe gibt,
    die Theorie und den Determinismus richtig zu verstehen.}

Damit ist nicht gemeint, dass es \enquote{gute}
oder erklärbare Gründe gibt, wieso Dinge passieren, oder dass diese Gründe irgendeine \enquote{Bedeutung}
hätten. Es heißt nur, dass es \emph{irgendwelche} Gründe gibt.\footnote{Wer zu viel Langeweile hat und mal zwei bis drei Stunden bequatscht werden möchte, kann Johannes ja mal fragen, wieso aus dieser Annahme folgt, dass es keinen freien Willen gibt.}

Eine wichtige Konsequenz des Prinzips von Ursache und Wirkung ist eine Art \emph{Transitivität von Realismus}: Wenn ich ein Ding als real erkannt habe, dann kann ich schlussfolgern, dass auch die Ursache dieses Dings real sein muss, selbst dann, wenn die Ursache sich meiner direkten Wahrnehmung entzieht. Dies kann ich auch über mehrere Zwischenschritte tun.

\subsubsection{Materialismus - Die Welt besteht nur aus Energie und Materie}

Nachdem die vorherigen Annahmen im Wesentlichen unangefochten geglaubt werden, ist diese Annahme etwas weniger populär
(aber insgesamt immer noch sehr populär) unter Wissenschaftlern.

Materialismus, manchmal auch \enquote{Physikalismus} genannt, bezeichnet die Idee, dass alles existierende
\enquote{materiell}
ist. Heutzutage meint man damit, dass alles existierende entweder eine Form von Energie oder Materie ist.\footnote{Und Physiker würden sogar sagen, dass Materie nur eine spezielle Form von Energie ist} Und würde die Physik noch eine dritte
Kategorie erfinden, die weder Materie noch Energie ist, aber trotzdem \emph{physikalisch}
ist, dann wäre auch das damit gemeint.\cite{sep-physicalism}

Diese Idee ist nicht so einschränkend, wie sie zunächst klingt, und ist trotzdem sehr weit gefasst (auch wenn Kritiker
das oft absichtlich falsch verstehen). Insbesondere können Dinge wie Bewusstsein, Gefühle, und alles, was sonst gerne
als Gegenbeispiel aufgeführt wird, als physische Prozesse verstanden werden.\footnote{Das Bewusstsein ist ein Punkt,
    an dem selbst ernstzunehmende Wissenschaftler und Philosophen manchmal mit dem Materialismus hadern. Es gibt durchaus Debatten,
    ob man diese Annahme aufgeben muss im Angesichts der großen Fragen,
    die das Phänomen Bewusstsein aufzuwerfen scheint. Die große Mehrheit der Wissenschaftler sieht hingegen keine Lücke
    in den Grundlagen, sondern nur eine (hoffentlich temporäre) Lücke in unseren Messmethoden und Theorien.

    Full disclosure: Johannes ist strenger Materialist und ist fest davon überzeugt,
    dass auch das Bewusstsein eine streng materialistische Erklärung hat.}

Was diese Idee ausschließt, sind Mechanismen von Ursache und Wirkung, die sich unserer (selbst indirekten)
Beobachtung entziehen. Ein guter Wissenschaftler ist z.\,B. im Prinzip durchaus bereit, daran zu glauben, dass Seelen und Geister
existieren, aber es wird erwartet, dass das, was die Seelen bzw. die Geister tun, bitteschön in einer
\emph{physikalischen}
Weise getan wird, dass es einen Wirkmechanismus gibt, der beschreibt, wie die nicht-materielle Seele mit der Materie des
Körpers interagiert, dass es Spuren gibt, die dieser Wirkmechanismus in der materiellen Welt hinterlässt etc. Und wenn es das gibt, dann gibt es keinen Grund, die Seele nicht auch als \enquote{materiell} zu bezeichnen. Sie wäre dann halt eine zuvor unbekannte Form von Materie oder Energie.

Kurzum: Die Antwort \enquote{Frag' nicht! Ist halt Magie.} wird einfach nicht akzeptiert.

\subsubsection{Reduktionismus - Alles lässt sich auf die Grundbestandteile der Welt zurückführen}

Auch dies ist eine Annahme, die von einer Minderheit der Wissenschaftler (teilweise) in Frage gestellt wird. Sie ist eng verwandt mit dem Materialismus und der Begriff wird manchmal austauschbar verwendet.

Unter Reduktionismus versteht man die Annahme, dass Objekte und Prozesse auf einer \enquote{höheren Ebene} eben nicht
\enquote{mehr sind als die Summe ihrer Teile}
, sondern dass sich auch die komplexesten Objekte und Prozesse, die existieren, sich stets auf die fundamentalen
Bestandteile der Wirklichkeit reduzieren lassen, zumindest im Prinzip.

Niemand verlangt ernsthaft, dass es \emph{praktisch}
möglich ist, z.\,B. alle Dinge, die in einem Fußballstadion am Spieltag geschehen, mittels Quarks, Elektronen und der
elektromagnetischen Feldtheorie herzuleiten, aber der allgemein verbreitete Glaube ist, dass dies zumindest
\emph{im Prinzip} möglich ist: Aus der Teilchenphysik folgt die statistische Physik (Thermodynamik, Fluiddynamik, etc.) und die Chemie, aus der Chemie und der Thermodynamik folgt die Biochemie, aus der Biochemie folgt die Biologie, aus der Biologie die Medizin etc.

Diese Annahme ist eng verbunden mit der Annahme des Materialismus. Jemand, der an Seelen glaubt und daran, dass diese das menschliche Verhalten beeinflussen, ist i.\,d.\,R. geneigt, beides abzulehnen und stattdessen zu glauben, dass Seelen nicht-materiell sind, ihre Wirkung nicht-physikalisch und dass menschliches Verhalten mehr ist als Biochemie und Psychologie.

\subsubsection{Vollständigkeit}

Aus den vorherigen Annahmen lässt sich letztlich ableiten, dass die experimentellen Methoden der modernen Wissenschaft (siehe nächstes Kapitel) uns erlaubt, tatsächliche Informationen über die Welt herauszufinden.

Materialismus und Reduktionismus sind bei genauer Betrachtung auch \emph{fast} dazu geeignet, diesen Schluss umzukehren und zu sagen: Alles, was existiert ist auch prinzipiell beobachtbar.

Aber eben nur fast. Zum einen sind da natürlich praktische Hürden. Nur weil etwas \emph{im Prinzip} beobachtbar ist, heißt das nicht, dass wir hier und heute das Wissen und die Technologie besitzen, es auch tatsächlich zu beobachten, oder dass wir diese je besitzen werden. Prinzipiell beobachtbar könnte heißen, dass für eine solche Beobachtung in der Praxis mehr Energie als eine explodierende Sonne nötig wäre und mehr Platz als eine Galaxie.

Zum anderen benötigen wir noch eine weitere metaphysische Annahme. Bisher haben wir \emph{hinreichende} Bedingungen für Realität gesammelt: Unsere Sinneswahrnehmungen sind (abzüglich optischer Täuschungen \& Co) hinreichend, um etwas als real zu erkennen. Das Prinzip von Ursache \& Wirkung lässt uns von unseren direkten Sinneneindrücken auch auf die Realität andere Dinge schlussfolgern.

Aber bisher sagt uns nichts, dass etwas, das real ist, auch immer irgendwelche Konsequenzen haben muss, die man direkt oder indirekt beobachten könnte.

Diese Annahme müssen wir nicht treffen. Und viele Menschen (insbesondere Nicht-Wissenschaftler) wollen das auch nicht. Wir könnten uns bereit erklären, stets die Option des nicht-messbaren zuzulassen.

Beim genaueren Nachdenken muss man sich jedoch die Frage stellen, was es denn überhaupt bedeuten soll, \enquote{real} zu sein, aber keine beobachtbaren Interaktionen mit der Welt zu haben, nicht einmal im Prinzip. Das hieße, dass ein solches Ding niemals irgendetwas bewegen würde, niemals sichtbar wäre, niemals fühlbar, niemals irgendeine Reaktion oder Erfahrung hervorrufen würde, niemals irgendetwas verändern würde, selbst über sehr viele indirekte Schritte nicht. In diesem Sinne wäre ein solches Ding das uninteressanteste Objekt im ganzen Universum; zwar \enquote{real}, aber komplett irrelevant für alles und jeden, überall und für immer.

Welchen Vorteil hat es aber dann überhaupt, ein solches Ding als \enquote{real} zu bezeichnen? Die Existenz oder Nicht-Existenz dieses Dings wäre eine rein sprachliche Konvention, die uns das Kommunizieren schwer macht, aber sonst keinerlei Auswirkungen hätte.

Aus diesem Grund, bietet es sich zur sprachlichen Vereinfachung an, solche Dinge einfach pauschal als nicht-existent zu deklarieren.
\pagebreak


\section{What could possibly go wrong?!}
\input{biases}
\subsection{RCTs -- Der \enquote{Goldstandard}}

Für medizinische Studien, die die Wirksamkeit von Medikamenten und Behandlungsmethoden überprüfen, hat sich über die Jahrzehnte ein \enquote{Goldstandard} etabliert, wie eine solche Studie idealerweise auszusehen hat, um maximale Aussagekraft zu besitzen und möglichst viele Biases auszuschließen:
\begin{enumerate}
\item Randomized: Studienteilnehmer werden zufällig in Gruppen eingeteilt. Es wird keine Vorauswahl oder Vorsortierung nach irgendwelchen Kriterien getroffen.
\item Controlled: Es gibt mindestens zwei solche Gruppen -- Eine Gruppe, die das Medikament bzw. die Behandlung erhält, und mind. eine \emph{Kontrollgruppe}, die beispielweise ein Placebo bzw. eine Scheinbehandlung erhält.
\item Double-blind: Weder die Teilnehmer, noch die Durchführenden wissen, wer welcher Gruppe zugeordnet wurde. Praktisch bedeutet das oft, dass während der Studie die Gruppen nur A,B,C,... heißen und erst nach Abschluss der Auswertung aufgedeckt wird, welches davon die Placebo-Gruppe(n) waren.
\end{enumerate}
In den letzten Jahren setzt sich mehr und mehr durch, dass dieser Standard um einen weiteren Punkt zu erweitern ist:
\begin{enumerate}[resume]
\item Pre-registered: Studien werden vor Beginn öffentlich vorgeanmeldet. Nicht nur, dass sie stattfinden, sondern auch der genaue Studienablauf, die beabsichtigte Anzahl von Studienteilnehmern und Auswertungsmethodik der Daten werden vorher festgelegt. Die Veröffentlichung der Studienergebnisse schließt einen Vergleich zwischen Voranmeldung und tatsächlichem Studienverlauf ein.
\end{enumerate}

Es gibt ähnliche Idealvorstellungen für andere Studien. Beispiel: Wenn gegen eine Krankheit bereits andere Medikamente etabliert sind, will man i.d.R. nicht nur wissen, \emph{ob} das neue Medikament wirkt, sondern ob es \emph{besser} wirkt als die bisherigen Optionen. In so einem Fall würde man als (ggf. zusätzliche) Kontrolle die Behandlung nach aktuellem Stand betrachten. Manchmal wird auch eine gemischte Gruppe betrachtet, welche die alte \emph{und} neue Behandlung erhält. Dies kann sinnvoll sein, wenn es sehr unterschiedliche Behandlungsansätze sind, die auf unterschiedlichen Wirkmechanismen beruhen und daher plausibel ist, dass sie gemeinsam wesentlich besser wirken als jede für sich genommen.

Jedes der genannten Kriterien soll gegen bestimmte Formen von Bias schützen

\subsubsection{Randomized controlled trials}

Eine Kontrollgruppe ist notwendig, um überhaupt eine Überprüfung irgendeiner Hypothese durchzuführen. Ein empirischer Test einer Hypothese erwartet zwangsläufig einen beobachtbaren Unterschied zwischen den Optionen \enquote{Hypothese ist wahr} und \emph{Hypothese ist falsch}. Wenn die Hypothese also ist, dass ein Medikament wirksam gegen eine Krankheit ist, muss man eine Gruppe von Patienten, die das Medikament erhalten, vergleichen mit einer Gruppe von Patienten, die das nicht tun. Und nur, wenn die Patienten, die es erhalten, messbar bessere 

\medskip
Randomisierung der Studienteilnehmer soll u.A. gegen Selection-Bias schützen und den offensichtlichsten Fehlern, die man machen könnte. Es wäre z.B. ein sehr schlechtes Studiendesign, alle jungen Menschen in die Behandlungsgruppe zu stecken und dann zu schlussfolgern, dass die Behandlung besonders erfolgreich ist, da man aufgrund dieser Vorauswahl nicht unterscheiden kann, ob die Jugend oder die Behandlung für den Erfolg verantwortlich waren. Junge Menschen sind einfach generell besser darin, sich von Krankheiten zu erholen.

Natürlich kann eine Randomisierung \emph{während} der Studie nicht davor schützen, dass eine Selektion getroffen wird, wer überhaupt an der Studie teilnimmt. Beispielsweise will man oftmals ganz gezielt diejenigen Patienten betrachtet, die die fragliche Krankheit auch tatsächlich haben.

\subsubsection{Blinded \& double blinded trials}

Der Faktor Mensch ist nicht zu unterschätzen und alleine die Kenntnis, wer das Placebo erhält, verzerrt potentiell das Ergebnis.

\medskip
Und zwar möglicherweise in beide Richtungen: Patienten, die wissen, dass sie das echte Medikament erhalten, können alleine durch diese Erkenntnis eine größere Wirkung bei sich wahrnehmen. Patienten können aber auch einem \enquote{Nocebo}-Effekt unterliegen und sich durch die Kenntnis, dass sie das echte Medikament erhalten, unter \enquote{Erfolgsdruck} gesetzt fühlen. Der zusätzliche Stress kann zu einer messbaren Verminderung des Behandlungserfolgs führen.

Eine Placebo-Kontrollgruppe ist genau dafür da, um diese pyschosomatische Wirkung von der biochemischen Wirkung des Medikaments zu separieren.

Natürlich ist es zutreffend, dass jede spätere echte Behandlung mit dem neuen Medikament ebenfalls einen Placebo- oder Nocebo-Effekt beinhalten wird. Dennoch ist es wichtig, dass man diesen Effekt in Wirksamkeitsstudien sauber separiert. Zum einen natürlich, weil er nicht Gegenstand der Studie ist. Die Frage, die es zu beantworten gilt, ist ja, ob/wie gut \emph{das Medikament} wirkt, nicht wie groß der Placebo-Effekt gerade ist. Der Placebo- oder Nocebo-Effekt würde schließlich bei jeder anderen kleinen, weißen Pille (mutmaßlich) genauso auftreten und man will ja die \emph{darüber hinausgehende} Wirkung der Behandlung maximieren. Dazu kommt, dass selbst das beste Studiendesign nicht verhindern wird, dass die Patienten besondere Aufmerksamkeit erhalten, eben weil sie Studienteilnehmer sind. Alleine diese zusätzliche Aufmerksamkeit verstärkt den Placebo-Effekt dramatisch, was einen großen Unterschied zwischen der Behandlung als Teil der Studie und der späteren Routine-Behandlung ausmacht. Der \enquote{reale} Placebo-Effekt kann einfach nicht realistisch Teil einer Studie sein.

Placebo- oder Nocebo-Effekte sind zudem stark von der Situation und der Krankheit abhängig. Insbesondere dort, wo die Symptome der Krankheit sowieso stark mit der menschlichen Psyche interagieren, ist er besonders stark. Überspitzt formuliert ist es unmöglich, mit dem Placebo-Effekt einen gebrochenen Arm zu heilen, aber extrem plausibel, Schmerzen, Übelkeit oder emotionale Zustände mittels Placebo-Effekt zu verbessern.

\medskip
Auch die Durchführenden der Studie müssen enquote{blind} sein, um die Studie vor Confimation-Bias zu schützen. Wissen die Durchführenden der Studie, wer das Medikament und wer die Placebos erhält, werden sie (ggf. unbewusst!) die Symptome und Behandlungserfolge dieser Patienten unterschiedlich bewerten.

Man unterscheidet in diesem Sinne zwischen \enquote{einfach} und \enquote{doppelt blinden} Studien.

\subsubsection{Pre-Registration}

Die Praxis hat gezeigt, dass nicht nur kognitive Biases von den direkt an der Studie beteiligten Menschen Ergebnisse verzerren. Das wissenschaftliche System als Ganzes kann ebenfalls dazu beitragen, die Wahrheit zu verschleiern und zu verzerren. Konkret ist es heutzutage oft so, dass Unternehmen und Wissenschaftler, die Studien durchführen auf die eine oder andere Wiese, mehr oder weniger offensichtlich einem Erfolgsdruck ausgesetzt sind. In gewisser Weise ist das natürlich gewollt: Die Menschheit als Ganzes hat natürlich ein berechtigtes Interesse daran, dass kein Geld und Resourcen für unnütze Studien verschwendet werden. Jede einzelne Institution und jedes Unternehmen hat dies auch.

In der Praxis führt das aber auch zu ungewollten Artefakten. Im Extremfall werden Forschungsergebnisse bewusst und böswillig manipuliert. Weniger extrem können Forschungsergebnisse einfach unveröffentlicht bleiben, wenn sie als unvorteilhaft empfunden werden. Insbesondere negative und Nicht-Ergebnisse erreichen oft nie das Licht der Öffentlichkeit, da die Öffentlichkeit (ganz bewusst) erfolgreiche Forschung bevorzugt, fördert und belohnt. Und selbst Forscher mit besten Absichten können unbewusst Verzerrungen erzeugen, wenn sie ihre Daten einfach unvorsichtig analysieren.

\medskip
Selektives Nicht-Veröffentlichen von Studien ist eine Gefahr nicht für die einzelne Studie, aber für das Gesamtbild, das man vom aktuellen Stand der Wissenschaft erhält, wenn man nur die veröffentlichten Studien ansieht (was sonst sollte man ansehen!?).

Um es zu überspitzen: Wenn 1000 Studien die Effektivität von Homöopathie untersuchen, aber nur 10 davon veröffentlicht werden, weil sie ein positives Ergebnis haben, dann ist der Schluss, den die Öffentlichkeit aus den zur Verfügung stehenden Studienergebnissen ziehen kann, immer noch der falsche. Selbst wenn diese zehn Studien alle einstimmig sind, ist die Realität vermutlich eine andere als es der scheinbare Konsens der Wissenschaft in diesem Fall nahelegt.

Wenn man durch Prä-Registrierung hingegen weiß, dass 990 Studien ihre Ergebnisse gar nicht erst veröffentlicht haben, ist es für die Öffentlichkeit viel einfacher zu schlussfolgern, dass die 10 erfolgreichen Studien die Ausreißer sind und Homöopathie in Wirklichkeit wirkungslos ist (was tatsächlich der Fall ist).

\medskip
Nicht nur die Absicht und Themen von Studien voranzukündigen, sondern auch die beabsichtigte Methodik, hat einen weiteren Vorteil: Es verhindert Artefakte, die durch \enquote{$p$-Hacking}\footnote{Benannt nach dem Signifikanzniveau $p$, das als Bewertungskriterium dient, ob ein statistisches Ergebnis als \enquote{signifikant} eingeschätzt wird.} entstehen, eine gewisse Art  von (absichtliche oder unabsichtliche) fehlerhafte Analysen der zur Verfügung stehenden Daten.

\smallskip
Zwei extrem Beispiele: Wenn man mit demselben einen Datensatz immer und immer wieder verschiedene Hypothesen testet, dann wird man irgendwann rein zufällig eine Hypothese finden, die in diesem Datensatz bestätigt wird, denn ein endlicher Datensatz kann einfach nicht unendlich viel Entropie enthalten um gegenüber unendlich vielen potentiellen Hypothesen neutral zu erscheinen.

Vorher zu veröffentlichen, welche Analysen man durchführen wird, beugt dieser Art von Analysefehlern vor.

\smallskip
Ähnlich gelagert ist der Ansatz, dieselbe Hypothese mit einem sehr großen, aber variablen Datensatz zu testen. Wenn kein Effekt vorhanden ist und die Daten hinreichend viel Rauschen haben im Vergleich zu meiner Test-Variable, dann kommt es rein zufällig zu Abweichungen in die eine oder andere Richtung. Wenn also nur lange genug Daten gesammelt werden, können auch Abweichungen über dem Signifikanzniveau auftreten. Würde man weitere Daten sammeln, würde sich das langfristig wieder ausgleichen; bricht man die Studie aber in diesem Moment ab (irgendwann muss man ja aufhören, Daten zu sammeln), dann hat man ebenfalls ein signifikantes Ergebnis \enquote{Aus dem nichts} erzeugt.

Die Anzahl der Studienteilnehmer vorher festzulegen, beugt dem vor.

\smallskip
Prä-Registrierung eröffnet zudem die Möglichkeit, vor Beginn der Datenerfassung bereits das Studiendesign einem Peer-Review zu unterziehen und Feedback einzuholen, wie viele verschiedene Datensätze nötig wären, um die beabsichtigten Hypothesen zu überprüfen, wie groß diese Datensätze sein müssten im Vergleich zur erwarteten Größenordnung des Effekts uvm.

\pagebreak


\section{Tag Drei -- Fragen über Fragen}
\input{fragen}

\pagebreak
\printbibliography
\end{document}
