% !TeX root = wissenschaftstheorie.tex
% !TeX spellcheck = de_DE

\subsection{Metaphysik}

Metaphysik bezeichnet die philosophische Disziplin, die ganz allgemein nach der Grundstruktur der Realität fragt.\cite{sep-metaphysics}
Zur Metaphysik gehören Fragen wie \enquote{Leben wir in der Matrix?},
\enquote{Existiert überhaupt irgendetwas und, wenn ja, warum?}, \enquote{Gibt es Gott?}, \enquote{Gibt es freien Willen?} uvm.

Das Ziel aller Wissenschaft\footnote{Eigentlich aller Versuche der Erkenntnisgewinnung,
    welche die Menschheit je unternommen hat. Aber wir wollen bei unserem eigentlichen Thema bleiben.}
ist es, die Realität zu verstehen. Die Metaphysik hilft uns dabei, bei diesem Unterfangen nicht ganz von vorne anfangen
zu müssen, sondern stellt uns Begrifflichkeiten bereit, auf denen wir aufbauen können, um zu verstehen wie
Wissenschaftler (im Allgemeinen) die Welt sehen.

\medskip

Gewisse metaphysische Grundannahmen werden in der Wissenschaft normalerweise nicht hinterfragt und in allem Tun und
Denken axiomatisch vorausgesetzt. Wir wollen das auch tun, aber zumindest wollen wir ein paar dieser Annahmen, die sonst
implizit bleiben, jetzt explizit nennen:

\subsubsection{Realismus I - Die Welt existiert}

Dies steht im Gegensatz zur Idee des Solipsismus. Nach René Decartes'\footnote{franz. Philosoph und Mathematiker, 1596 –- 1650} berühmten \enquote{Ich denke, also bin ich}
-Argument, kann sich jeder von uns sicher sein, dass man selbst existiert, aber es ist schwer absolute Sicherheit
darüber zu gewinnen, dass noch irgendjemand oder irgendetwas anderes wirklich existiert\footnote
{Descartes war der Meinung,
    auch dafür wasserfeste Argumente gefunden zu haben; der Rest ist der Philosophen ist nicht so sehr davon
    überzeugt wie von \enquote{Ich denke, also bin ich}}. Solipsismus ist, stark vereinfacht, die Idee, dass alles, wirklich alles andere Einbildung des eigenen Geistes ist.
\footnote
{Das ist kein Standpunkt, den irgendjemand ernsthaft vertritt, sondern eher eine denkbare Alternative,
    die man eigentlich gerne ausschließen möchte, aber es mangels guter Argument nicht unbedingt kann.}

Die Wissenschaft lehnt dies ab und geht von einer real existierenden Welt aus, die es zu untersuchen gilt. Offenbar
eine sehr nützliche Annahme, weil sonst kein Untersuchungsgegenstand für Wissenschaftler existierte.

\subsubsection{Realismus II - Die Welt ist wahrnehmbar}

Dies ist im Wesentlichen die Annahme, dass wir nicht in der Matrix\footnote
{Wie haben Philosophen sich nur früher Dinge erklärt,
    bevor philosophische Ideen verfilmt wurden und allgemeines Kulturgut wurden!?}
leben: Die Welt existiert nicht nur; unsere Sinne nehmen das, was existiert, zumindest teilweise wahr.

Man beachte, dass niemand (mehr) postuliert, dass unsere Sinne eine fehlerfreie oder vollständige Erfassung der Welt
ermöglichen; ganz im Gegenteil. Aber es wird eben zumindest davon ausgegangen, dass in den allermeisten Fällen
Dinge, die wir sehen und anfassen können, auch tatsächlich existieren. Vielleicht existiert noch mehr als das, aber
\emph{mindestens} das existiert schon einmal.

\subsubsection{Realismus III - Die Welt ist \enquote{objektiv}}

Das Wort \enquote{objektiv}
hat viele (mehr oder weniger) verschiedene Bedeutungen in verschiedenen Kontexten. Hier meinen wir damit das, was
man auch als \enquote{Intersubjektivität}
bezeichnen könnte, also die Annahme, dass für alle anderen dieselbe Welt existiert, die für einen selbst existiert.
Anders formuliert könnte man dies auch als eine noch stärkere Ablehnung des Solipsismus verstehen: Ich lebe nicht in meiner
eigenen Welt, die nur für mich existiert; die Welt existiert auch unabhängig von mir.

Unter Objektivität kann man auch eine gewisse \enquote{Neutralität} der Welt gegenüber dem Wahrnehmenden verstehen: Wenn ich in der exakt gleichen Situation wäre wie du, dann würde ich exakt das wahrnehmen wie du. Nur, weil ich eben ich bin, gelten nicht plötzlich andere Regeln für mich als für dich.

\subsubsection{Regularität \& Kausalität - Die Natur folgt Gesetzen}

Dies ist teilweise eine überprüfbare, beobachtbare Behauptung, teilweise aber auch eine Grundannahme. Wir fangen mit
der Wissenschaft nicht von vorne an; wir wissen bereits etliches, was unsere Vorfahren herausgefunden haben und das
allermeiste davon folgt auch irgendwelchen Gesetzmäßigkeiten. Wissenschaftler haben darüber hinausgehend aber auch
die Annahme, dass alles andere ebenfalls gesetzmäßig geschieht. Möglicherweise kennen wir noch nicht alle
Naturgesetze, aber dass es sie gibt und dass alles, was passiert, von irgendeinem dieser Gesetze erfasst wird, sind
Annahmen, die (fast) alle Wissenschaftler voraussetzen.

Darunter fällt auch die Annahme, dass Naturgesetze sich nicht spontan ändern, sondern räumlich und zeitlich konstant sind. (Und
wenn sie es einmal nicht sind, dann ist die Änderung ein Ausdruck eines noch grundlegenderen Naturgesetzes).

Noch etwas grundlegender ist darin auch die Annahme versteckt, dass die Natur dem Prinzip von Ursache und Wirkung folgt:
Nichts geschieht einfach so; alles hat irgendeine Ursache.\footnote{Nein,
    wir diskutieren jetzt nicht über Determinismus und Quantenphysik. So viel Zeit haben wir nicht. Es sei nur so viel
    gesagt, dass die Quantenmechanik auch eine deterministische Theorie ist, wenn man sich Mühe gibt,
    die Theorie und den Determinismus richtig zu verstehen.}

Damit ist nicht gemeint, dass es \enquote{gute}
oder erklärbare Gründe gibt, wieso Dinge passieren, oder dass diese Gründe irgendeine \enquote{Bedeutung}
hätten. Es heißt nur, dass es \emph{irgendwelche} Gründe gibt.\footnote{Wer zu viel Langeweile hat und mal zwei bis drei Stunden bequatscht werden möchte, kann Johannes ja mal fragen, wieso aus dieser Annahme folgt, dass es keinen freien Willen gibt.}

Eine wichtige Konsequenz des Prinzips von Ursache und Wirkung ist eine Art \emph{Transitivität von Realismus}: Wenn ich ein Ding als real erkannt habe, dann kann ich schlussfolgern, dass auch die Ursache dieses Dings real sein muss, selbst dann, wenn die Ursache sich meiner direkten Wahrnehmung entzieht. Dies kann ich auch über mehrere Zwischenschritte tun.

\subsubsection{Materialismus - Die Welt besteht nur aus Energie und Materie}

Nachdem die vorherigen Annahmen im Wesentlichen unangefochten geglaubt werden, ist diese Annahme etwas weniger populär
(aber insgesamt immer noch sehr populär) unter Wissenschaftlern.

Materialismus, manchmal auch \enquote{Physikalismus} genannt, bezeichnet die Idee, dass alles existierende
\enquote{materiell}
ist. Heutzutage meint man damit, dass alles existierende entweder eine Form von Energie oder Materie ist.\footnote{Und Physiker würden sogar sagen, dass Materie nur eine spezielle Form von Energie ist} Und würde die Physik noch eine dritte
Kategorie erfinden, die weder Materie noch Energie ist, aber trotzdem \emph{physikalisch}
ist, dann wäre auch das damit gemeint.\cite{sep-physicalism}

Diese Idee ist nicht so einschränkend, wie sie zunächst klingt, und ist trotzdem sehr weit gefasst (auch wenn Kritiker
das oft absichtlich falsch verstehen). Insbesondere können Dinge wie Bewusstsein, Gefühle, und alles, was sonst gerne
als Gegenbeispiel aufgeführt wird, als physische Prozesse verstanden werden.\footnote{Das Bewusstsein ist ein Punkt,
    an dem selbst ernstzunehmende Wissenschaftler und Philosophen manchmal mit dem Materialismus hadern. Es gibt durchaus Debatten,
    ob man diese Annahme aufgeben muss im Angesichts der großen Fragen,
    die das Phänomen Bewusstsein aufzuwerfen scheint. Die große Mehrheit der Wissenschaftler sieht hingegen keine Lücke
    in den Grundlagen, sondern nur eine (hoffentlich temporäre) Lücke in unseren Messmethoden und Theorien.

    Full disclosure: Johannes ist strenger Materialist und ist fest davon überzeugt,
    dass auch das Bewusstsein eine streng materialistische Erklärung hat.}

Was diese Idee ausschließt, sind Mechanismen von Ursache und Wirkung, die sich unserer (selbst indirekten)
Beobachtung entziehen. Ein guter Wissenschaftler ist z.\,B. im Prinzip durchaus bereit, daran zu glauben, dass Seelen und Geister
existieren, aber es wird erwartet, dass das, was die Seelen bzw. die Geister tun, bitteschön in einer
\emph{physikalischen}
Weise getan wird, dass es einen Wirkmechanismus gibt, der beschreibt, wie die nicht-materielle Seele mit der Materie des
Körpers interagiert, dass es Spuren gibt, die dieser Wirkmechanismus in der materiellen Welt hinterlässt etc. Und wenn es das gibt, dann gibt es keinen Grund, die Seele nicht auch als \enquote{materiell} zu bezeichnen. Sie wäre dann halt eine zuvor unbekannte Form von Materie oder Energie.

Kurzum: Die Antwort \enquote{Frag' nicht! Ist halt Magie.} wird einfach nicht akzeptiert.

\subsubsection{Reduktionismus - Alles lässt sich auf die Grundbestandteile der Welt zurückführen}

Auch dies ist eine Annahme, die von einer Minderheit der Wissenschaftler (teilweise) in Frage gestellt wird. Sie ist eng verwandt mit dem Materialismus und der Begriff wird manchmal austauschbar verwendet.

Unter Reduktionismus versteht man die Annahme, dass Objekte und Prozesse auf einer \enquote{höheren Ebene} eben nicht
\enquote{mehr sind als die Summe ihrer Teile}
, sondern dass sich auch die komplexesten Objekte und Prozesse, die existieren, sich stets auf die fundamentalen
Bestandteile der Wirklichkeit reduzieren lassen, zumindest im Prinzip.

Niemand verlangt ernsthaft, dass es \emph{praktisch}
möglich ist, z.\,B. alle Dinge, die in einem Fußballstadion am Spieltag geschehen, mittels Quarks, Elektronen und der
elektromagnetischen Feldtheorie herzuleiten, aber der allgemein verbreitete Glaube ist, dass dies zumindest
\emph{im Prinzip} möglich ist: Aus der Teilchenphysik folgt die statistische Physik (Thermodynamik, Fluiddynamik, etc.) und die Chemie, aus der Chemie und der Thermodynamik folgt die Biochemie, aus der Biochemie folgt die Biologie, aus der Biologie die Medizin etc.

Diese Annahme ist eng verbunden mit der Annahme des Materialismus. Jemand, der an Seelen glaubt und daran, dass diese das menschliche Verhalten beeinflussen, ist i.\,d.\,R. geneigt, beides abzulehnen und stattdessen zu glauben, dass Seelen nicht-materiell sind, ihre Wirkung nicht-physikalisch und dass menschliches Verhalten mehr ist als Biochemie und Psychologie.

\subsubsection{Vollständigkeit}

Aus den vorherigen Annahmen lässt sich letztlich ableiten, dass die experimentellen Methoden der modernen Wissenschaft (siehe nächstes Kapitel) uns erlaubt, tatsächliche Informationen über die Welt herauszufinden.

Materialismus und Reduktionismus sind bei genauer Betrachtung auch \emph{fast} dazu geeignet, diesen Schluss umzukehren und zu sagen: Alles, was existiert ist auch prinzipiell beobachtbar.

Aber eben nur fast. Zum einen sind da natürlich praktische Hürden. Nur weil etwas \emph{im Prinzip} beobachtbar ist, heißt das nicht, dass wir hier und heute das Wissen und die Technologie besitzen, es auch tatsächlich zu beobachten, oder dass wir diese je besitzen werden. Prinzipiell beobachtbar könnte heißen, dass für eine solche Beobachtung in der Praxis mehr Energie als eine explodierende Sonne nötig wäre und mehr Platz als eine Galaxie.

Zum anderen benötigen wir noch eine weitere metaphysische Annahme. Bisher haben wir \emph{hinreichende} Bedingungen für Realität gesammelt: Unsere Sinneswahrnehmungen sind (abzüglich optischer Täuschungen \& Co) hinreichend, um etwas als real zu erkennen. Das Prinzip von Ursache \& Wirkung lässt uns von unseren direkten Sinneneindrücken auch auf die Realität andere Dinge schlussfolgern.

Aber bisher sagt uns nichts, dass etwas, das real ist, auch immer irgendwelche Konsequenzen haben muss, die man direkt oder indirekt beobachten könnte.

Diese Annahme müssen wir nicht treffen. Und viele Menschen (insbesondere Nicht-Wissenschaftler) wollen das auch nicht. Wir könnten uns bereit erklären, stets die Option des nicht-messbaren zuzulassen.

Beim genaueren Nachdenken muss man sich jedoch die Frage stellen, was es denn überhaupt bedeuten soll, \enquote{real} zu sein, aber keine beobachtbaren Interaktionen mit der Welt zu haben, nicht einmal im Prinzip. Das hieße, dass ein solches Ding niemals irgendetwas bewegen würde, niemals sichtbar wäre, niemals fühlbar, niemals irgendeine Reaktion oder Erfahrung hervorrufen würde, niemals irgendetwas verändern würde, selbst über sehr viele indirekte Schritte nicht. In diesem Sinne wäre ein solches Ding das uninteressanteste Objekt im ganzen Universum; zwar \enquote{real}, aber komplett irrelevant für alles und jeden, überall und für immer.

Welchen Vorteil hat es aber dann überhaupt, ein solches Ding als \enquote{real} zu bezeichnen? Die Existenz oder Nicht-Existenz dieses Dings wäre eine rein sprachliche Konvention, die uns das Kommunizieren schwer macht, aber sonst keinerlei Auswirkungen hätte.

Aus diesem Grund, bietet es sich zur sprachlichen Vereinfachung an, solche Dinge einfach pauschal als nicht-existent zu deklarieren.