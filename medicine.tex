\subsection{RCTs -- Der \enquote{Goldstandard}}

Für medizinische Studien, die die Wirksamkeit von Medikamenten und Behandlungsmethoden überprüfen, hat sich über die Jahrzehnte ein \enquote{Goldstandard} etabliert, wie eine solche Studie idealerweise auszusehen hat, um maximale Aussagekraft zu besitzen und möglichst viele Biases auszuschließen:
\begin{enumerate}
\item Randomized: Studienteilnehmer werden zufällig in Gruppen eingeteilt. Es wird keine Vorauswahl oder Vorsortierung nach irgendwelchen Kriterien getroffen.
\item Controlled: Es gibt mindestens zwei solche Gruppen -- Eine Gruppe, die das Medikament bzw. die Behandlung erhält, und mind. eine \emph{Kontrollgruppe}, die beispielweise ein Placebo bzw. eine Scheinbehandlung erhält.
\item Double-blind: Weder die Teilnehmer, noch die Durchführenden wissen, wer welcher Gruppe zugeordnet wurde. Praktisch bedeutet das oft, dass während der Studie die Gruppen nur A,B,C,... heißen und erst nach Abschluss der Auswertung aufgedeckt wird, welches davon die Placebo-Gruppe(n) waren.
\end{enumerate}
In den letzten Jahren setzt sich mehr und mehr durch, dass dieser Standard um einen weiteren Punkt zu erweitern ist:
\begin{enumerate}[resume]
\item Pre-registered: Studien werden vor Beginn öffentlich vorgeanmeldet. Nicht nur, dass sie stattfinden, sondern auch der genaue Studienablauf, die beabsichtigte Anzahl von Studienteilnehmern und Auswertungsmethodik der Daten werden vorher festgelegt. Die Veröffentlichung der Studienergebnisse schließt einen Vergleich zwischen Voranmeldung und tatsächlichem Studienverlauf ein.
\end{enumerate}

Es gibt ähnliche Idealvorstellungen für andere Studien. Beispiel: Wenn gegen eine Krankheit bereits andere Medikamente etabliert sind, will man i.d.R. nicht nur wissen, \emph{ob} das neue Medikament wirkt, sondern ob es \emph{besser} wirkt als die bisherigen Optionen. In so einem Fall würde man als (ggf. zusätzliche) Kontrolle die Behandlung nach aktuellem Stand betrachten. Manchmal wird auch eine gemischte Gruppe betrachtet, welche die alte \emph{und} neue Behandlung erhält. Dies kann sinnvoll sein, wenn es sehr unterschiedliche Behandlungsansätze sind, die auf unterschiedlichen Wirkmechanismen beruhen und daher plausibel ist, dass sie gemeinsam wesentlich besser wirken als jede für sich genommen.

Jedes der genannten Kriterien soll gegen bestimmte Formen von Bias schützen

\subsubsection{Randomized controlled trials}

Eine Kontrollgruppe ist notwendig, um überhaupt eine Überprüfung irgendeiner Hypothese durchzuführen. Ein empirischer Test einer Hypothese erwartet zwangsläufig einen beobachtbaren Unterschied zwischen den Optionen \enquote{Hypothese ist wahr} und \emph{Hypothese ist falsch}. Wenn die Hypothese also ist, dass ein Medikament wirksam gegen eine Krankheit ist, muss man eine Gruppe von Patienten, die das Medikament erhalten, vergleichen mit einer Gruppe von Patienten, die das nicht tun. Und nur, wenn die Patienten, die es erhalten, messbar bessere 

\medskip
Randomisierung der Studienteilnehmer soll u.A. gegen Selection-Bias schützen und den offensichtlichsten Fehlern, die man machen könnte. Es wäre z.B. ein sehr schlechtes Studiendesign, alle jungen Menschen in die Behandlungsgruppe zu stecken und dann zu schlussfolgern, dass die Behandlung besonders erfolgreich ist, da man aufgrund dieser Vorauswahl nicht unterscheiden kann, ob die Jugend oder die Behandlung für den Erfolg verantwortlich waren. Junge Menschen sind einfach generell besser darin, sich von Krankheiten zu erholen.

Natürlich kann eine Randomisierung \emph{während} der Studie nicht davor schützen, dass eine Selektion getroffen wird, wer überhaupt an der Studie teilnimmt. Beispielsweise will man oftmals ganz gezielt diejenigen Patienten betrachtet, die die fragliche Krankheit auch tatsächlich haben.

\subsubsection{Blinded \& double blinded trials}

Der Faktor Mensch ist nicht zu unterschätzen und alleine die Kenntnis, wer das Placebo erhält, verzerrt potentiell das Ergebnis.

\medskip
Und zwar möglicherweise in beide Richtungen: Patienten, die wissen, dass sie das echte Medikament erhalten, können alleine durch diese Erkenntnis eine größere Wirkung bei sich wahrnehmen. Patienten können aber auch einem \enquote{Nocebo}-Effekt unterliegen und sich durch die Kenntnis, dass sie das echte Medikament erhalten, unter \enquote{Erfolgsdruck} gesetzt fühlen. Der zusätzliche Stress kann zu einer messbaren Verminderung des Behandlungserfolgs führen.

Eine Placebo-Kontrollgruppe ist genau dafür da, um diese pyschosomatische Wirkung von der biochemischen Wirkung des Medikaments zu separieren.

Natürlich ist es zutreffend, dass jede spätere echte Behandlung mit dem neuen Medikament ebenfalls einen Placebo- oder Nocebo-Effekt beinhalten wird. Dennoch ist es wichtig, dass man diesen Effekt in Wirksamkeitsstudien sauber separiert. Zum einen natürlich, weil er nicht Gegenstand der Studie ist. Die Frage, die es zu beantworten gilt, ist ja, ob/wie gut \emph{das Medikament} wirkt, nicht wie groß der Placebo-Effekt gerade ist. Der Placebo- oder Nocebo-Effekt würde schließlich bei jeder anderen kleinen, weißen Pille (mutmaßlich) genauso auftreten und man will ja die \emph{darüber hinausgehende} Wirkung der Behandlung maximieren. Dazu kommt, dass selbst das beste Studiendesign nicht verhindern wird, dass die Patienten besondere Aufmerksamkeit erhalten, eben weil sie Studienteilnehmer sind. Alleine diese zusätzliche Aufmerksamkeit verstärkt den Placebo-Effekt dramatisch, was einen großen Unterschied zwischen der Behandlung als Teil der Studie und der späteren Routine-Behandlung ausmacht. Der \enquote{reale} Placebo-Effekt kann einfach nicht realistisch Teil einer Studie sein.

Placebo- oder Nocebo-Effekte sind zudem stark von der Situation und der Krankheit abhängig. Insbesondere dort, wo die Symptome der Krankheit sowieso stark mit der menschlichen Psyche interagieren, ist er besonders stark. Überspitzt formuliert ist es unmöglich, mit dem Placebo-Effekt einen gebrochenen Arm zu heilen, aber extrem plausibel, Schmerzen, Übelkeit oder emotionale Zustände mittels Placebo-Effekt zu verbessern.

\medskip
Auch die Durchführenden der Studie müssen enquote{blind} sein, um die Studie vor Confimation-Bias zu schützen. Wissen die Durchführenden der Studie, wer das Medikament und wer die Placebos erhält, werden sie (ggf. unbewusst!) die Symptome und Behandlungserfolge dieser Patienten unterschiedlich bewerten.

Man unterscheidet in diesem Sinne zwischen \enquote{einfach} und \enquote{doppelt blinden} Studien.

\subsubsection{Pre-Registration}

Die Praxis hat gezeigt, dass nicht nur kognitive Biases von den direkt an der Studie beteiligten Menschen Ergebnisse verzerren. Das wissenschaftliche System als Ganzes kann ebenfalls dazu beitragen, die Wahrheit zu verschleiern und zu verzerren. Konkret ist es heutzutage oft so, dass Unternehmen und Wissenschaftler, die Studien durchführen auf die eine oder andere Wiese, mehr oder weniger offensichtlich einem Erfolgsdruck ausgesetzt sind. In gewisser Weise ist das natürlich gewollt: Die Menschheit als Ganzes hat natürlich ein berechtigtes Interesse daran, dass kein Geld und Resourcen für unnütze Studien verschwendet werden. Jede einzelne Institution und jedes Unternehmen hat dies auch.

In der Praxis führt das aber auch zu ungewollten Artefakten. Im Extremfall werden Forschungsergebnisse bewusst und böswillig manipuliert. Weniger extrem können Forschungsergebnisse einfach unveröffentlicht bleiben, wenn sie als unvorteilhaft empfunden werden. Insbesondere negative und Nicht-Ergebnisse erreichen oft nie das Licht der Öffentlichkeit, da die Öffentlichkeit (ganz bewusst) erfolgreiche Forschung bevorzugt, fördert und belohnt. Und selbst Forscher mit besten Absichten können unbewusst Verzerrungen erzeugen, wenn sie ihre Daten einfach unvorsichtig analysieren.

\medskip
Selektives Nicht-Veröffentlichen von Studien ist eine Gefahr nicht für die einzelne Studie, aber für das Gesamtbild, das man vom aktuellen Stand der Wissenschaft erhält, wenn man nur die veröffentlichten Studien ansieht (was sonst sollte man ansehen!?).

Um es zu überspitzen: Wenn 1000 Studien die Effektivität von Homöopathie untersuchen, aber nur 10 davon veröffentlicht werden, weil sie ein positives Ergebnis haben, dann ist der Schluss, den die Öffentlichkeit aus den zur Verfügung stehenden Studienergebnissen ziehen kann, immer noch der falsche. Selbst wenn diese zehn Studien alle einstimmig sind, ist die Realität vermutlich eine andere als es der scheinbare Konsens der Wissenschaft in diesem Fall nahelegt.

Wenn man durch Prä-Registrierung hingegen weiß, dass 990 Studien ihre Ergebnisse gar nicht erst veröffentlicht haben, ist es für die Öffentlichkeit viel einfacher zu schlussfolgern, dass die 10 erfolgreichen Studien die Ausreißer sind und Homöopathie in Wirklichkeit wirkungslos ist (was tatsächlich der Fall ist).

\medskip
Nicht nur die Absicht und Themen von Studien voranzukündigen, sondern auch die beabsichtigte Methodik, hat einen weiteren Vorteil: Es verhindert Artefakte, die durch \enquote{$p$-Hacking}\footnote{Benannt nach dem Signifikanzniveau $p$, das als Bewertungskriterium dient, ob ein statistisches Ergebnis als \enquote{signifikant} eingeschätzt wird.} entstehen, eine gewisse Art  von (absichtliche oder unabsichtliche) fehlerhafte Analysen der zur Verfügung stehenden Daten.

\smallskip
Zwei extrem Beispiele: Wenn man mit demselben einen Datensatz immer und immer wieder verschiedene Hypothesen testet, dann wird man irgendwann rein zufällig eine Hypothese finden, die in diesem Datensatz bestätigt wird, denn ein endlicher Datensatz kann einfach nicht unendlich viel Entropie enthalten um gegenüber unendlich vielen potentiellen Hypothesen neutral zu erscheinen.

Vorher zu veröffentlichen, welche Analysen man durchführen wird, beugt dieser Art von Analysefehlern vor.

\smallskip
Ähnlich gelagert ist der Ansatz, dieselbe Hypothese mit einem sehr großen, aber variablen Datensatz zu testen. Wenn kein Effekt vorhanden ist und die Daten hinreichend viel Rauschen haben im Vergleich zu meiner Test-Variable, dann kommt es rein zufällig zu Abweichungen in die eine oder andere Richtung. Wenn also nur lange genug Daten gesammelt werden, können auch Abweichungen über dem Signifikanzniveau auftreten. Würde man weitere Daten sammeln, würde sich das langfristig wieder ausgleichen; bricht man die Studie aber in diesem Moment ab (irgendwann muss man ja aufhören, Daten zu sammeln), dann hat man ebenfalls ein signifikantes Ergebnis \enquote{Aus dem nichts} erzeugt.

Die Anzahl der Studienteilnehmer vorher festzulegen, beugt dem vor.

\smallskip
Prä-Registrierung eröffnet zudem die Möglichkeit, vor Beginn der Datenerfassung bereits das Studiendesign einem Peer-Review zu unterziehen und Feedback einzuholen, wie viele verschiedene Datensätze nötig wären, um die beabsichtigten Hypothesen zu überprüfen, wie groß diese Datensätze sein müssten im Vergleich zur erwarteten Größenordnung des Effekts uvm.